

\subsubsection{Nature et objectifs du jeu d’échecs}

\begin{enumerate}

\item Le jeu d’échecs se joue entre deux adversa
ires  qui  déplacent  alternativement  des  pièces  sur  un  plateau 
carré appelé 
"échiquier"
. Le joueur ayant les pièces blanches commence la partie. On dit qu’un joueur 
"a le trait"
lorsque le coup de son adversaire a été 
"joué"
.
\item L’objectif  de
chaque  joueur  est  de  placer  le  roi  adverse 
"sous  une  attaque"
de  telle  manière  que 
l’adversaire n’ait aucun coup légal. On dit que le joueur qui atteint ce but a 
"maté"
le roi adverse et 
gagné la partie. Laisser son roi sous une attaque, exposer son roi à
une attaque et aussi 
"prendre"
le 
roi adverse n'est pas autorisé. L’adversaire dont le roi a été maté a perdu la partie.

\item Si la position est telle qu’aucun des deux joueurs n’a la possibilité de mater, la partie est nulle. \\

\end{enumerate}

\subsubsection{Mouvement et position des pièces}

\begin{figure}
  \includegraphics[scale=0.5]{Screenshot from 2016-04-27 17:17:56.png}
  \caption{Position initiale des pièces}
\end{figure}

\begin{figure}
  \includegraphics[]{Screenshot from 2016-04-27 17:11:55.png}
  \caption{mouvement de la reine}
  \end{figure}

\begin{figure}
  \includegraphics[width=\linewidth]{Screenshot from 2016-04-27 17:12:07.png}
  \caption{Mouvement du cheval}
  \end{figure}

\begin{figure}
  \includegraphics[width=\linewidth]{Screenshot from 2016-04-27 17:22:11.png}
  \caption{Mouvement du fou}
  \end{figure}

